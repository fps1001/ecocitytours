\capitulo{1}{Introducción}

El crecimiento de población en las ciudades \cite{nieuwenhuijsen_urban_2020} y el turismo como catalizador de la gentrificación supone el gran campo de batalla para gobiernos locales de los países occidentales que han visto como la falta de una legislación controlada del turismo supone un grave problema afectando múltiples niveles de la convivencia, economía y el medio ambiente. A pesar de los avances en la promoción de un nuevo modelo urbano, muchas urbes aún enfrentan desafíos significativos en la integración de prácticas sostenibles en la vida cotidiana de sus habitantes. La falta de información accesible y personalizada sobre rutas y actividades que promuevan la movilidad sostenible y el turismo responsable es un marco común que se debe desarrollar si se quiere evitar que el conflicto crezca sin fin. Esta brecha de información impide que tanto residentes como turistas adopten hábitos más sostenibles que beneficien a la comunidad local y al medio ambiente en un marco global.

Fomentar el Turismo Sostenible supone una gran oportunidad para intentar contrarrestar la deriva actual. Y es que el turismo es un motor fundamental de la economía a nivel global y por tanto tiene la capacidad de transformarse para ayudar a la sostenibilidad del planeta.  Destaca en este marco de trabajo el  objetivo \acrshort{ods} número 11 titulado \textit{Ciudades y Comunidades Sostenibles}. Según el informe de la \textit{UNESCO}, \cite{ionescu_progress_2024}, no solo es crucial por sí mismo, sino que actúa como un factor multiplicador, influyendo indirectamente en la consecución de otros \acrshort{ods} debido a su enfoque integral y transversal.

Eco City Tours \textbf{aúna estos esfuerzos al proporcionar una herramienta práctica y accesible} para la promoción del ODS11 y la movilidad sostenible. La aplicación ha sido desarrollada en Flutter y utiliza \acrfull{llm} para generar rutas turísticas personalizadas que conecten \acrlong{pdi}. La aplicación se enfoca en las \textbf{preferencias del usuario}, ofreciendo \textbf{rutas optimizadas} para ciclistas y peatones promoviendo así la movilidad sostenible.

Las preferencias del usuario se comunicarán al modelo a través de un menú, donde éste podrá elegir qué lugar visitar, si realizar el tour a pie o en bicicleta, cuántos \acrlong{pdi} incluir en la ruta y sus gustos a la hora de viajar. El sistema generará el camino más corto, calculado entre los distintos \acrshort{pdi} a visitar, usando para ello un servicio de geolocalización.

Mientras que muchas de las aplicaciones similares revisadas solo utilizan información comercial para definir rutas turísticas, Eco City Tours solicita que se tengan en cuenta prácticas sostenibles como la deslocalización del turismo a la hora de elegir destinos. Todas estas consideraciones enriquecen la experiencia turística de los visitantes \cite{mitas_tell_2023}, e incluso promueven el crecimiento económico de las comunidades locales. De este modo, Eco City Tours logra un impacto positivo tanto a nivel local como global.